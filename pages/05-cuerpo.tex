\chapter{Introducción}

    \section{Motivación}
        
        La Agencia Digital de Andalucía (ADA) presentó en 2021 un plan de inversión para los años 2022-2025 que se centra en varias áreas estratégicas de desarrollo tecnológico, incluyendo la ciberseguridad; esta inversión del Gobierno andaluz no solo impulsará la creación de un Centro de Ciberseguridad que coordinará la Estrategia Andaluza de Ciberseguridad 2022-2025, sino que además, pretende conseguir que revierta en el desarrollo económico de la región, que ya ha empezado a aceptar empresas como Google, que ha instalado su Centro de Excelencia de Ciberseguridad en Málaga.
        
        Bajo el paraguas de este desarrollo tecnológico, el aumento de la demanda de profesionales en este campo ha despertado un gran interés en los estudiantes por aprender sobre la ciberseguridad.
        
        Sin embargo, muchos de ellos se enfrentan a la dificultad de encontrar recursos educativos completos y accesibles debido a las restricciones que imponen algunas plataformas de entrenamiento, donde los recursos que ofrecen tienen barreras que dificultan el acceso a la formación, incluyendo la necesidad de suscripciones para acceder al contenido completo.
        
        Esto puede desmotivar a los estudiantes y reducir su capacidad para aprender y desarrollar habilidades en el campo de la ciberseguridad.
    
    
    \section{Objetivos}
        \label{sec:objetivos}
    
        Se busca desarrollar de una plataforma web que sea ligera, gratuita, \textit{open-source} y fácilmente extensible, con el objetivo principal de simplificar la introducción de futuros estudiantes en el sector de la ciberseguridad, con un enfoque específico en el pentesting.
        
        La plataforma proporcionará información documentada y sintetizada junto con entornos de prueba virtualizados para facilitar el aprendizaje mediante la experimentación.
        
        A diferencia de otras plataformas como Hack The Box y similares, el enfoque de esta plataforma no estará en la resolución de retos, como serían las pruebas de tipo CTF más habituales en el sector, sino en la construcción de entornos virtuales diseñados específicamente para experimentar y poner en práctica conceptos concretos.
        
        Esta herramienta brindará a los estudiantes una experiencia educativa gratuita y accesible en ciberseguridad, permitiéndoles aprender y desarrollar habilidades prácticas de manera efectiva y en línea con las necesidades del mercado.
    
    
    \section{Metodología}
    
        Se ha empleado un \textbf{desarrollo incremental} para la elaboración del proyecto: se partió de una base a la que se añadieron sucesivas mejoras sin que estas perjudicaran al resto del proyecto desarrollado hasta ese momento.
        
        El desarrollo de dichas mejoras se ha llevado a cabo utilizando el \textbf{sistema Kanban} de gestión de tareas, descomponiendo el proyecto en tareas lo más simples y atómicas posibles, distribuyendo los \textit{issues} por una serie de etapas conforme se iban realizando (\textit{Pendiente}, \textit{En progreso} y \textit{Terminado}), llevando a cabo un seguimiento de la evolución y el estado del proyecto.
        
        
        
        \newpage
    
    \section{Estructura global del proyecto}
        \label{sec:estructura-global}
        
        El proyecto se divide en tres pilares principales: la creación de una plataforma web, la investigación y selección de conceptos de pentesting y la virtualización de los entornos de prueba, resultando en el objetivo de este proyecto de trabajo de fin de grado, una plataforma web que proporcionará información sobre técnicas de pentesting y ofrecerá acceso a pruebas virtualizadas en forma de laboratorios. 
        
        \begin{figure}[h]
            \centering
            \includegraphics[scale=0.30]{images/Diagramas/Estructura global.png}
            \caption{Pilares fundamentales del proyecto}
            \label{fig:estructura-global}
        \end{figure}
            
        \subsection{Página web}
        
            La primera toma de contacto del usuario con la plataforma, la página web será la interfaz de usuario para acceder a los laboratorios y contenidos relacionados con la ciberseguridad.
            
            Respecto a su construcción, se han considerado dos opciones: programarla desde cero o utilizar un CMS (Sistema de Gestión de Contenidos); para la primera opción, se optaría por el uso del framework Astro, que permite crear sitios web rápidamente, además de proporcionar un amplio conjunto de herramientas para su personalización; por otro lado, se ha considerado el uso de WordPress, uno de los CMS más utilizados en el mundo y que cuenta con una amplia variedad de plantillas disponibles, lo que facilitaría el diseño de la interfaz.

            También se tuvieron en cuenta distintos constructores de páginas webs como Cargo Collective o Square Space.

            Finalmente, se decidió usar WordPress por su sencillez y su biblioteca de plugins, que permitiría integrar diferentes funcionalidades extra en lugar de programarlas manualmente como podría haber ocurrido con el uso de un framework.
                
            \subsubsection{Alojamiento}
            
                Respecto al alojamiento, se han considerado varias opciones, tales como GitHub Pages, Vercel, OVHcloud o Heroku.
                
                Sin embargo, al tratarse de una prueba de concepto, se ha decidido elaborar la infraestructura de forma local, lo que permitirá un mayor control y flexibilidad en el desarrollo, dando lugar a mejoras en futuras líneas de trabajo \ref{sec:futuras-lineas-trabajo}.

                En cuanto a la base de datos, se ha decidido utilizar MySQL por ser una opción muy utilizada, fácil de configurar y ser nativamente compatible con WordPress. La base de datos tendrá la menor cantidad de tablas posibles, almacenando únicamente la información necesaria sobre los usuarios registrados y sobre los laboratorios.
            
        \subsection{Investigación}
        
            Se centrará en recopilar y seleccionar diferentes conceptos y técnicas relacionados con el pentesting. Este apartado se verá mejor reflejado en la sección Catálogo \ref{sec:catalogo}.
        
        \subsection{Virtualización}

            La virtualización es una parte integral del proyecto, ya que permite la creación de varios laboratorios de prueba donde los usuarios aplicarán los conocimientos adquiridos en la plataforma; concretamente, se consideraron contenedores de Docker y máquinas virtuales para crear estos laboratorios.
            
            Elegir Docker como plataforma de contenedores es una buena opción porque ofrece una gestión de recursos eficiente y una generación de imágenes sencilla; por otro lado, Vagrant Cloud es una buena opción para crear máquinas virtuales, ya que ofrece una amplia gama de imágenes preconfiguradas y un entorno de desarrollo fácil de usar.
            
            Se pueden utilizar diversas herramientas para gestionar los contenedores, en función de las necesidades específicas de cada laboratorio; por ejemplo, Kubernetes es una buena herramienta para la gestión de contenedores a gran escala (clústers de contenedores). Además, Docker Compose es una opción conveniente para configurar y ejecutar aplicaciones de contenedores múltiples.
            
            Por otro lado, también se consideraron plataformas que brindan servicios de almacenamiento y distribución de imágenes, como DockerHub para imágenes de contenedores de Docker y Vagrant Cloud para máquinas virtuales preconfiguradas. Estas dos plataformas brindan enlaces de descargas con las que un usuario puede acceder a recursos previamente montados por su autor.

            \newpage

            
    \section{Tecnologías}
    
        Las siguientes tecnologías y herramientas se emplearon en el desarrollo del proyecto:
        
        \subsection{GitHub}
        
            GitHub es una plataforma de desarrollo colaborativo de software que ofrece alojamiento de código fuente, seguimiento de problemas y herramientas de colaboración para proyectos de programación.
            
            Esta plataforma es muy popular en la comunidad de desarrolladores, ya que les permite compartir su trabajo con otros usuarios, pudiendo ser clonado, modificado y actualizado por otros desarrolladores, permitiendo la colaboración y el intercambio de conocimientos; o mantenerlo privado para uso personal o empresarial.
            
            Además del alojamiento de código fuente, GitHub también ofrece herramientas de seguimiento de problemas, donde los desarrolladores pueden reportar y resolver errores, solicitar nuevas características y discutir mejoras en el software. Esto facilita la comunicación y la colaboración entre los usuarios, además de mejorar la calidad del software desarrollado.
            
            
        \subsection{WordPress}
            
            WordPress \cite{wordpress} es una plataforma de código abierto para la creación y gestión de sitios web que fue publicada en 2003 y desde entonces, se ha convertido en una de las plataformas más populares y utilizadas en todo el mundo, siendo especialmente popular para la creación de blogs, aunque también es utilizada para sitios web corporativos, tiendas en línea, portafolios, entre otros. 
            
            WordPress es utilizado por muchos grandes sitios web y empresas, como el blog de noticias TIME \cite{time-web} y el sitio web de la BBC America \cite{bbc-america-web}. También es utilizado por miles de bloggers y pequeñas empresas en todo el mundo debido a su facilidad de uso y capacidad de personalización.

            Una de las principales ventajas de WordPress es que es muy fácil de usar y no se requiere experiencia técnica previa para comenzar a utilizarlo. Otra ventaja de WordPress es su personalización gracias a la gran cantidad de temas y complementos (plugins) disponibles. Estos complementos permiten añadir nuevas funcionalidades a un sitio web, como formularios de contacto, galerías de imágenes, integración con redes sociales... evitando la necesidad de programar dichas funcionalidades de forma manual.
            
            Por otra parte, WordPress es muy \textit{SEO-friendly}, incluyendo características que ayudan a mejorar el posicionamiento en los motores de búsqueda, como la capacidad de editar títulos, descripciones y etiquetas de metadatos, así como la posibilidad de utilizar herramientas de optimización de motores de búsqueda (SEO).
            
            \newpage

            
        \subsection{Docker}
            \label{sec:docker}
        
            Docker es una plataforma de software que permite crear, implementar y administrar aplicaciones en contenedores, siendo un contenedor un entorno aislado y seguro que recoge una aplicación y todas sus dependencias, lo que permite que se ejecute sin problemas en cualquier entorno informático, independientemente de las diferencias entre los sistemas operativos o las configuraciones de hardware.
            
            \begin{figure}[h]
                \centering
                
                \begin{subfigure}[h]{\textwidth}
                    \includegraphics[width=\textwidth]{images/Diagramas/Esquema de Contenedores.png}
                    \caption{Arquitectura de los contenedores}
                    \label{fig:arquitectura-contenedores}
                \end{subfigure}
                
                \begin{subfigure}[h]{\textwidth}
                    \includegraphics[width=\textwidth]{images/Diagramas/Esquema de MVs.png}
                    \caption{Arquitectura de las máquinas virtuales}
                    \label{fig:arquitectura-maquinasvirtuales}
                \end{subfigure}
                
                \caption{Diferencias entre contenedores y las máquinas virtuales}
                \label{fig:contenedores-vs-maquinasvirtuales}
            \end{figure}
            
            Docker funciona con imágenes, que son plantillas o modelos que contienen todos los componentes necesarios para ejecutar una aplicación en un contenedor, incluidos el código fuente, las librerías, los archivos de configuración y las dependencias del sistema. Estas imágenes pueden ser descargadas desde un registro centralizado llamado Docker Hub o construirse localmente por los desarrolladores.
            
            Una vez obtenida una imagen, puede usarse Docker para crear un contenedor a partir de ella, lo que implica crear una instancia de aplicación y, por tanto, el entorno en el que se ejecuta; los contenedores de Docker se pueden transferir fácilmente entre diferentes sistemas y entornos, lo que facilita la implementación y el escalado de aplicaciones. Además, Docker proporciona herramientas de administración de contenedores, como la capacidad de iniciar, detener, reiniciar y eliminar contenedores, así como la capacidad de monitorear su uso y rendimiento.
            
            \newpage


        \subsection{LAMP}

            Este es un acrónimo que se utiliza comúnmente en el mundo de los servidores para describir un conjunto de tecnologías de código abierto que se utilizan para construir y ejecutar aplicaciones web dinámicas.

            \begin{table}[!htbp]
                  \centering
                  
                  \begin{tabular}{|>{\centering\arraybackslash}m{3cm}|>{\centering\arraybackslash}m{3cm}|>{\centering\arraybackslash}m{8cm}|}
                        \hline
                        \textbf{Componente} & \textbf{Tipo} & \textbf{Descripción} \\
                        \hline
                        \hline
                        \textbf{Linux} & Sistema Operativo & Utilizado ampliamente en servidores web debido a su fiabilidad, seguridad y flexibilidad. \\
                        \hline
                        \textbf{Apache} & Servidor Web & Alojamiento de sitios web y aplicaciones web. \\
                        \hline
                        \textbf{MySQL} & RDBMS & Almacenamiento y recuperación de información para aplicaciones web. \\
                        \hline
                        \textbf{PHP} & Lenguaje de Programación & Creación de sitios web y aplicaciones web dinámicas. Procesa la lógica de la aplicación en el servidor. \\
                        \hline
                  \end{tabular}

                  \caption{Componentes habituales de la pila de software LAMP}
                  \label{table:lamp}
            \end{table}
            
            Algunas veces, la P de LAMP es sustituida por Perl o Python.                  

            \cleardoublepage

    
     
\chapter{Estado del arte}
    \label{cap:estado-arte}

    Actualmente existen diversas plataformas web de entrenamiento en ciberseguridad que ofrecen laboratorios y desafíos para mejorar las habilidades de seguridad ofensiva y defensiva de los usuarios; dichas plataformas han experimentado un crecimiento en popularidad en los últimos años debido a la creciente demanda de profesionales en el campo de la seguridad informática y a la necesidad de mejorar las habilidades tanto de los estudiantes como de los profesionales en el campo.
    
    Estas plataformas varían en su enfoque y contenido, desde laboratorios que simulan entornos de la vida real hasta desafíos de explotación de vulnerabilidades, así como eventos de naturaleza más lúdica y competitiva como los retos de capturar la bandera, también conocidos como CTFs (\textit{Capture The Flag}).
    
    Algunas de estas plataformas han sido utilizadas en la educación, ya que permiten a los estudiantes practicar y aplicar conceptos teóricos de seguridad informática en un entorno práctico; y también pueden ser útiles para empresas y organizaciones, ya sea para evaluar las habilidades de seguridad de los empleados o para mejorar la seguridad de sus sistemas.
    
    Respecto al futuro, se espera que continúen creciendo en popularidad y que aumente la variedad de laboratorios y desafíos que ofrecen, siendo muy probable que se produzca una mayor integración con otras tecnologías de inteligencia artificial, quizás dando lugar a una mejora en la calidad de esos desafíos y laboratorios mencionados, para que proporcionen una experiencia de usuario más personalizada.
    
    A continuación se presenta una breve descripción de algunas de las plataformas más populares actualmente, donde se muestran sus similitudes y sus diferencias:
    
    \newpage
    
    
    \section{Hack The Box}
    
        Plataforma web de entrenamiento que ofrece más de 200 desafíos y laboratorios de amplia variedad, diseñados para mejorar las habilidades de seguridad ofensiva y defensiva de los usuarios. Hack The Box se ha vuelto \textbf{la plataforma más popular} en la comunidad de seguridad informática debido a su \textbf{enfoque en la calidad de los desafíos y laboratorios} y su \textbf{comunidad activa de usuarios}.
        
        Sus retos abarcan desde desafíos básicos de hacking hasta laboratorios avanzados, y también ofrece una función llamada \textit{Boxes} diseñada para simular entornos de la vida real, como redes empresariales, que contienen múltiples vulnerabilidades que los usuarios pueden explotar para ganar acceso a sistemas y obtener información confidencial.
        
        \begin{figure}[h]
            \centering
            \includegraphics[width=\textwidth]{images/Capturas/Web de HTB.png}
            \caption{Web de Hack The Box}
            \label{fig:HTB-web}
        \end{figure}
        
        Una de las características más interesantes de esta plataforma es su enfoque en el hacking ético, ya que se fomenta una cultura de hacking responsable y legal y requiere que los usuarios acepten un código de conducta antes de unirse. Los usuarios también son alentados a informar sobre cualquier vulnerabilidad que encuentren en la plataforma.
        
        Hack the Box ofrece diferentes \textbf{planes de suscripción} para los usuarios interesados en acceder a su contenido:
        
        \begin{itemize}
            \item \textbf{Plan gratuito}: acceso limitado a una selección de desafíos y laboratorios, pero no incluye acceso a los \textit{boxes}.
        
            \item \textbf{Plan VIP}: acceso completo a la plataforma (todos los laboratorios, desafíos y \textit{boxes} disponibles).
        
            \item \textbf{Planes empresariales}: personalizados para empresas y organizaciones que deseen utilizar la plataforma para la formación y evaluación de sus empleados en seguridad informática.
        \end{itemize}
        
        \newpage

    
    \subsection{HTB Academy}
    
        Esta es una iniciativa de Hack The Box que también ofrece formación, pero al contrario que Hack The Box, centrada en la práctica y el desafío en tiempo real, HTB Academy \textbf{se enfoca en la enseñanza práctica de habilidades a través de cursos y laboratorios}.
        
        \begin{figure}[h]
            \centering
            \includegraphics[width=\textwidth]{images/Capturas/Web de HTB Academy.png}
            \caption{Web de HTB Academy}
            \label{fig:HTB-Academy-web}
        \end{figure}
        
        Los cursos están diseñados para ser prácticos, con una orientación en la experimentación activa que permite a los estudiantes adquirir habilidades en la práctica, y no solo a través de la teoría; cubren una amplia gama de temas, desde los fundamentos de la seguridad informática hasta temas avanzados como el análisis de malware, el hacking web y la ingeniería inversa, ya que están diseñados para ser accesibles desde principiantes hasta expertos en el sector.
        
        Esta plataforma cuenta con los mismos tipos de \textbf{planes de suscripción} que Hack The Box: gratuito, VIP y empresarial; aunque es importante destacar que tanto Hack The Box como HTB Academy son servicios diferentes, por lo que \textbf{los planes de ambas plataformas son completamente independientes entre sí}.
        
        \newpage
    
    
    \section{TryHackMe}
    
        Plataforma de aprendizaje de ciberseguridad basada en la experimentación activa, que proporciona una variedad de entornos de laboratorio virtuales y desafíos prácticos para ayudar a los usuarios a aprender y mejorar sus habilidades en seguridad informática; su contenido se encuentra organizado en \textbf{diferentes rutas de aprendizaje} que permiten a los usuarios desarrollar su conocimiento de forma estructurada.
        
        \begin{figure}[h]
            \centering
            \includegraphics[width=\textwidth]{images/Capturas/Web de THM.png}
            \caption{Web de TryHackMe}
            \label{fig:THM-web}
        \end{figure}
        
        La plataforma también cuenta con una comunidad activa y una función de \textit{gamificación} que proporciona una experiencia de aprendizaje más interactiva y entretenida, permitiendo que los usuarios puedan competir entre ellos, ganando puntos y recompensas por completar desafíos y resolver problemas de seguridad.
        
        TryHackMe ofrece diferentes \textbf{planes de suscripción} para los usuarios interesados:
        
        \begin{itemize}
            \item \textbf{Plan gratuito}: acceso limitado a una selección de desafíos y laboratorios.
        
            \item \textbf{Plan premium}: acceso completo a la plataforma (todos los laboratorios y desafíos).
        
            \item \textbf{Planes empresariales}: personalizados para empresas y organizaciones que deseen utilizar la plataforma para la formación y evaluación de sus empleados en seguridad informática.
        \end{itemize}
        
        \newpage
    
    
    \section{VulnHub}
    
        Plataforma de laboratorios que proporciona una gran cantidad de \textbf{máquinas virtuales vulnerables que los usuarios pueden descargar} y configurar en sus propios entornos de laboratorio para luego explotar sus vulnerabilidades; cada máquina virtual cuenta con descripción detallada de su objetivo y una guía paso a paso para ayudar a los estudiantes en su proceso de aprendizaje.
        
        \begin{figure}[h]
            \centering
            \includegraphics[width=\textwidth]{images/Capturas/Web de VulnHub.png}
            \caption{Web de VulnHub}
            \label{fig:VulnHub-web}
        \end{figure}
        
        Estos desafíos son diseñados por la comunidad, y la plataforma también ofrece la opción de que los usuarios puedan crear y compartir sus propios desafíos y máquinas virtuales.
        
        Al contrario de lo que sucede con otras plataformas, entre ellas Hack The Box y TryHackMe mencionadas anteriormente, \textbf{esta plataforma es completamente gratuita} y todo su contenido está construido por y para los usuarios.
        
        \newpage
    
    
    \section{OverTheWire}
    
        Plataforma de laboratorios \textbf{diseñada de manera progresiva}, donde los usuarios pueden avanzar en su aprendizaje de forma gradual, comenzando con los niveles más fáciles y avanzando hacia los más complejos; cada nivel de desafío presenta un objetivo diferente, haciendo que los usuarios deban usar su ingenio y habilidades en seguridad informática para resolver los desafíos y avanzar al siguiente nivel.
        
        \begin{figure}[h]
            \centering
            \includegraphics[width=\textwidth]{images/Capturas/Web de OverTheWire.png}
            \caption{Web de OverTheWire}
            \label{fig:OverTheWire-web}
        \end{figure}
        
        Uno de los aspectos únicos de OverTheWire es que los desafíos están diseñados para simular situaciones del mundo real, lo que permite a los estudiantes adquirir habilidades prácticas y relevantes para el mundo laboral de la ciberseguridad; pudiendo aplicar lo aprendido a situaciones reales y utilizar sus habilidades para asegurar sistemas y aplicaciones.
        
        Al contrario de lo que sucede con otras plataformas, entre ellas Hack The Box y TryHackMe mencionadas anteriormente, \textbf{esta plataforma es completamente gratuita} y todo su contenido está construido por y para los usuarios.
        
        \cleardoublepage
    
    

\chapter{Investigación previa}
    \label{cap:investigacion-previa}

    \section{Proceso creativo}
        \label{sec:proceso-creativo}

        Antes de comenzar a desarrollar el proyecto, resulta importante saber de forma más específica \textit{qué} se quiere llevar a cabo, y una vez decidio el objetivo, analizar las posibles opciones para conseguirlo.
        
        Ya se realizó una investigación previa sobre las diferentes plataformas de aprendizaje de ciberseguridad existentes -descritas en el Estado del Arte \ref{cap:estado-arte}- con el objetivo de analizar sus características y funcionalidades.
        
        La mayoría -y más conocidas- de estas plataformas son CTFs (Capture The Flag), y aunque son muy útiles para poner en práctica los conocimientos adquiridos, no suelen ofrecer un entorno de aprendizaje guiado, sino que el usuario debe resolver los desafíos por su cuenta, sin ningún tipo de documentación que le ayude a resolverlos, más allá de las pistas que ofrezca la propia plataforma o las soluciones que hayan publicado otros usuarios en Internet.

        Sin embargo, esto no siempre es así. Como se vió en el apartado anterior: Try Hack Me \cite{tryhackme} ofrece tanto retos guiados e instructivos, como retos puramente prácticos y competitivos; mientras que Hack The Box \cite{hackthebox}, al ser la plataforma de CTFs prácticos más reconocida, ha sido capaz de lanzar su propia plataforma educativa con retos guiados: HTB Academy \cite{hackthebox-academy}. 

        \subsection{Plataforma de CTFs}

            La primera idea planteada fue, precisamente, replicar una plataforma de CTFs (Capture The Flag). Esta versión del proyecto se centraría más en enseñar al usuario distintos conceptos de la ciberseguridad, acompañados de una serie de retos que pondrían a prueba los conocimientos adquiridos; algo parecido a las plataformas existentes mencionadas en el apartado anterior.

            Esta idea fue descartada porque ya existen muchas plataformas de CTFs, y la voluntad con este proyecto era de hacer algo distinto. Se requería un entorno de aprendizaje guiado y como se mencionó anteriormente, los CTFs no suelen ofrecerlo.

        
        \subsection{Plataforma como portal de descargas}

            Tomando como ejemplo los retos descargables de la Academia Hacker de Incibe \cite{retos-INCIBE}, se planteó la posibilidad de crear máquinas virtuales y alojarlas en el servidor a modo de elementos descargables; es decir, la plataforma contaría con un sitio web que ofrecería información documentada sobre distintos conceptos de ciberseguridad, pero la virtualización se llevaría a cabo por el propio usuario en su equipo, experimentando localmente.

            Esta idea también fue descartada porque no solo plantea el mismo problema de fricción descrito en la idea anterior, sino que además plantea otros problemas relacionados con el espacio de almacenamiento (tanto en el servidor como en el equipo del usuario) y con la necesidad del usuario de gestionar máquinas virtuales.
            
            No existiría la posibilidad de ofrecer un entorno de trabajo en el que el usuario pudiera interactuar con los laboratorios de forma remota y si además, este debe descargar y configurar un hypervisor, o instalar Docker para acceder a los entornos de prueba en su equipo, se perdería esa esencia de minimalismo y sencillez que debería ofrecer la plataforma.

            Sin embargo, cabe destacar que existen herramientas que podrían ser de utilidad para este proyecto, como: Vagrant \cite{vagrant}, que permite automatizar la creación y configuración de máquinas virtuales de forma declarativa; y Vagrant Cloud \cite{vagrant-cloud}, que es un repositorio de máquinas virtuales ya configuradas para instalar usando \texttt{vagrant}. También puede ser útil incluir en la plataforma de una guía introductoria a la configuración de una máquina virtual para tests de intrusión (pentesing) con Kali Linux, ya que si bien no es objeto de este proyecto que los usuarios aprendan sobre virtualización, no se puede ignorar el hecho de que es necesario el uso de máquinas virtuales para el sector de la ciberseguridad.

        
        \subsection{Plataforma de laboratorios}

            Por último, se planteó crear una plataforma de laboratorios con la que el usuario pudiera acceder a distintos entornos de forma remota, sin necesidad de descargar ni configurar nada en su equipo. La plataforma estaría compuesta de un sitio web donde estaría recogida la documentación, mientras que el servidor alojaría los entornos virtualizados asociados a dicha documentación.
            
            La principal diferencia y rasgo a destacar de este planteamiento es que los laboratorios actuarían como \textit{sandboxes}; es decir, aunque un laboratorio estaría planteado para que el usuario pudiera resolver un problema descrito, también ofrecería al usuario contar con un entorno con herramientas preinstaladas en el caso de querer hacer sus propias pruebas sobre un concepto en específico.

            Esta idea fue la que más se acercaba a los objetivos del proyecto, y por tanto fue la que se escogió para su desarrollo: solventaba las fricciones de las ideas anteriores, y además ofrecía un entorno de aprendizaje guiado y práctico, que era el objetivo principal del proyecto.

            \newpage

        \subsection{Sobre Jupyter}

            \subsubsection{Jupyter Notebooks}
                
                Jupyter \cite{jupyter} es un proyecto \textit{open-source} que permite crear documentos interactivos llamados \textit{notebooks} que contienen código ejecutable, ecuaciones, visualizaciones y texto explicativo que sigue un formato de Markdown enriquecido y estructurado en bloques. Estos \textit{notebooks} están basadas en JSON y se pueden compartir con otros usuarios, por lo que se convierten en una herramienta muy útil para la investigación y el análisis de datos, pero sobre todo para la colaboración y el aprendizaje.
            
                Debido a la naturaleza de este proyecto, se planteó la idea de incluir páginas en la plataforma que fueran Jupyter Notebooks, de esta forma el usuario podría acceder a la página que contendría documentación y código anidado (característica principal de Jupyter), y en ella podría insertar y ejecutar código, así como generar distintos recursos de forma interactiva para que el usuario pudiera experimentar con ella.

                Sin embargo, se descartó la idea ya que no constituía una prioridad en el desarrollo del proyecto y sus casos de uso eran muy limitados, puesto que el código verdaderamente importante sería ejecutado en las \textit{sandboxes}.
                

            \subsubsection{JupyterHub}

                JupyterHub \cite{jupyterhub} es una aplicación que permite crear un entorno de Jupyter Notebooks en un servidor remoto, de forma que los usuarios puedan acceder a él a través de un navegador web. Esta aplicación es muy útil para la colaboración y el aprendizaje, ya que permite a los usuarios compartir documentos y ejecutar código en un entorno seguro y controlado.
                
                Siguiendo con lo mencionado en el apartado anterior, se planteó la idea de usar JupyterHub para crear distintos entornos de Jupyter Notebook en la nube para que los usuarios pudieran acceder a un entorno de Jupyter Notebooks en el que pudieran ejecutar código y experimentar con los recursos generados por el sistema, en lugar de integrarlos en la plataforma.

                Esta idea también se descartó por ser ser una versión más compleja de la anterior.
                
                \newpage

                
        \subsection{Sobre Docker}

            Docker, descrito en la sección \ref{sec:docker}, permitiría la creación de contenedores para los distintos laboratorios.
                
            Esta idea surgió a partir del artículo \textit{Cyber Security Testbeds and Malware Testing} \cite{securitylab-malware-analysis} de la Universidad de Trento, donde un grupo de investigación realizaban un estudio sobre los efectos de un \textit{exploit} de una aplicación sobre varias versiones de la misma y usada en distintos entornos de desarrollo.

            \begin{quotation}
                
                Given a software environment $E$, an exploit $X$ that successfully subverts an application $A$, that is running on $E$:
                
                \begin{itemize}
                    \item Will $X$ be successful on an application $A$ running on another environment $E'$?
                    \item Will $X$ be successful on another version of $A$, $A'$, running on $E$?
                    \item Will $X$ be successful on another version of $A$, $A'$, running on $E'$?
                \end{itemize}

            \end{quotation}

            Para ello, el grupo de investigación desarrolló un conjunto de pruebas a partir de contenedores, usando combinaciones de un \textbf{conenedor principal} junto a \textbf{contenedores secundarios}.

            \begin{quote}

                Instead of creating separate virtual machines for every application/configuration we use the Linux Containers technology that provides virtualization capabilities on the operation system level. \[...\] We use Docker to implement two types of containers: (1) software-specific that contain operating system, webserver and database engine; (2) application-specific that is build on top of a desired software-specific container and also encapsulates the application files. The figure on the right shows an example Wordpress3.2 application container that has been built on top of the “ubuntu-apache-mysql” software container.

            \end{quote}

            Y la figura que representa dicho planteamiento es la siguiente:

            \begin{figure}[htbp]
                \centering
                \includegraphics[scale=0.75]{images/Diagramas/Articulo.png}
                \caption{Contenedor de Wordpress3.2 sobre contenedor con Ubuntu + Apache + MySQL}
                \label{fig:articulo}
            \end{figure}

            \newpage
            

            \subsubsection{Red de contenedores de Docker}

                Inicialmente se interpretó el sistema del artículo anterior como una red de contenedores.
                
                Por tanto, en lugar de crear un contenedor para cada \textit{sandbox}, se planteó la idea de crear una red contenedores para cada laboratorio, de forma que cada red estuviera compuesta por aquellos elementos a tratar en dicho laboratorio, y que a su vez sería accesible desde el exterior a través de un puerto específico, permitiendo a un usuario conectarse a través de SSH.

                Esto podría lograrse usando Docker Compose \cite{docker-compose}, que permite crear y ejecutar aplicaciones Docker de forma sencilla, usando un archivo YAML que describe los servicios que componen la aplicación, así como las dependencias entre ellos.

                Visto en perspectiva, queda bastante claro que el mecanismo del proyecto planteado en el artículo y el propuesto en esta sección no son similares más allá del uso de contenedores.
                
                
            \subsubsection{Dependencia de contenedores de Docker}

                Por tanto, el planteamiento inicial de usar una red de contenedores era errónea.
                
                La estructura descrita en el artículo se forma a través del uso de la tecnología de contenedores de Linux, alojando cada aplicación y configuración de software en un contenedor.

                Una posible réplica de ese mecanismo podría realizarse de la siguiente forma:

                \begin{enumerate}

                    \item \textbf{Instalar Docker en el sistema operativo}
                    
                    Realizable siguiendo las instrucciones de instalación en el sitio web oficial de Docker.
                
                    \item \textbf{Crear imágenes específicas del software}
                    
                    Realizable utilizando un archivo Dockerfile que describa la configuración y los componentes que deben incluirse en dicha imagen.

                    \item \textbf{Crear un contenedor específico de la aplicación}

                    Para crear este tipo de contenedor, se debe construir sobre un contenedor específico del software previamente creado; para ello, se deben encapsular los archivos de la aplicación dentro del contenedor específico del software.

                    \item \textbf{Ejecutar contenedores}
                    
                    Una vez creados los contenedores específicos del software y de la aplicación, pueden ejecutarse con el comando \texttt{docker run} de Docker.

                \end{enumerate}

                En el paso 2 se crea una imagen específica del software a partir de un archivo Dockerfile que describe la configuración y los componentes que deben incluirse en los contenedores resultantes de dicha imagen.

                En el paso 3, se crea una imagen específica de la aplicación que se construye sobre la imagen del software previamente creada; es decir, se crea una nueva imagen que constituye una capa de configuración adicional sobre la configuración de la propia imagen del software, que actua como base. Esta nueva imagen encapsula los archivos de la aplicación y se crea también a partir de un archivo Dockerfile.

                El resultado final es un contenedor específico de la aplicación que reutiliza el sistema operativo y los componentes incluidos en la imagen del software, pero que añade los archivos específicos para el funcionamiento de la aplicación.

                A continuación se muestran 2 pseudo-códigos que representan el resultado final descrito:

                \begin{figure}[!htbp]
                    \centering
                    
                    \begin{subfigure}[!htbp]{\textwidth}
                        \centering
                        \includegraphics[width=0.8\textwidth]{images/Contenedor A.png}
                        \caption{Dockerfile de una imagen principal (padre)}
                    \end{subfigure}
                    
                    \begin{subfigure}[!htbp]{\textwidth}
                        \centering
                        \includegraphics[width=0.8\textwidth]{images/Contenedor B.png}
                        \caption{Dockerfile de una imagen secundaria (hija)}
                    \end{subfigure}
                    
                    \caption{Dependencia de contenedores}
                    \label{fig:dependencia-contenedores}
                \end{figure}
                
                El Dockerfile de la imagen hija hace referencia a la imagen padre mediante la línea $\texttt{FROM software\_container}$, estableciendo la dependencia. Además, se copian los archivos de la aplicación en la imagen y se establece como el directorio de trabajo. Finalmente, se ejecuta la aplicación con la línea $\verb|CMD ["python", "app.py"]|$, lo que mantendrá activos los contenedores resultantes de dicha imagen.

                \begin{figure}[!htbp]
                    \centering
                    \includegraphics[scale=0.14]{images/Diagramas/Contenedor A.png}
                    \caption{Creación de contenedores principales (padres)}
                    \label{fig:contenedor-padre}
                \end{figure}
                
                \begin{figure}[!htbp]
                    \centering
                    \includegraphics[scale=0.14]{images/Diagramas/Contenedor B.png}
                    \caption{Creación de contenedores secundarios (hijos)}
                    \label{fig:contenedor-hijo}
                \end{figure}
                
                \newpage


    
    \section{Análisis de Tecnologías}
        
        \subsection{Página web}

            \begin{table}[!htbp]
                \centering
                
                \small
                
                \begin{tabular}{|>{\centering\arraybackslash}m{3cm}|>{\centering\arraybackslash}m{3.5cm}|>{\centering\arraybackslash}m{3.5cm}|>{\centering\arraybackslash}m{3.5cm}|}
                    \hline
                    \textbf{Características} & \textbf{WordPress} & \textbf{Astro} & \textbf{Drupal} \\
                    \hline
                    \hline
                    \textbf{Tipo de plataforma} & CMS & Framework & CMS \\
                    \hline
                    \textbf{Orientación} & Sitios web: pequeños y medianos & Aplicaciones web & Sitios web: grandes y complejos\\
                    \hline
                    \textbf{Lenguaje de programación} & PHP & Javascript & PHP \\
                    \hline
                    \textbf{Facilidad de uso} & Simple: no requiere experiencia técnica ni programación & Compleja: requiere experiencia técnica y en programación & Compleja: requiere experiencia con el propio CMS \\
                    \hline
                    \textbf{Personalización} & Alta: temas y plugins & Muy alta: desarrollo web & Alta: módulos y extensiones \\
                    \hline
                    \textbf{Escalabilidad} & Sí: sitios web pequeños y medianos & Sí: aplicaciones web pequeñas y grandes & Sí: sitios grandes y complejos \\
                    \hline
                    \textbf{Comunidad} & Grande: usuarios y desarrolladores & Creciente: desarrolladores & Mediana: usuarios y desarrolladores, muy dedicados y comprometidos \\
                    \hline
                    \textbf{Popularidad} & El CMS más popular & Un framework relativamente nuevo & Uno de los CMS más usados \\
                    \hline
                    \textbf{Seguridad} & Alta: pero es vulnerable a ataques por su popularidad & Variable: depende de la implementación del desarrollador & Alta: CMS centrado en la seguridad y la protección \\
                    \hline
                \end{tabular}
                
                \caption{Comparativa entre WordPress, Astro y Drupal}
                \label{tab:wordpress-vs-astro-vs-drupal}
            \end{table}

            Se ha elegido WordPress como la plataforma para la creación de la página web del proyecto debido a sus ventajas en cuanto a facilidad de uso, personalización y escalabilidad, que eran los principales aspectos a tener en cuenta para el desarrollo.

            
            \subsubsection{WordPress y Astro}

            \paragraph{Facilidad de uso}
                
                Factor importante en la decisión de elegir esta herramienta ya que no se cuentan con conocimientos previos de programación web ni a nivel técnico, por lo que debía usarse una herramienta que no tuviera en cuenta esas necesidades.
                
                WordPress parece cubrir ese aspecto de forma notable debido a que su popularidad se basa, precisamente, en la cantidad de usuarios que son capaces de crear una página web sin necesidad de ser desarrolladores o contar con experiencia previa.
            
            \paragraph{Personalización}
            
                También fue considerada un factor clave ya que se contaba con la creación de una página web atractiva y funcional.
                
                Teniendo en cuenta lo mencionado en el punto anterior acerca de la falta de conocimientos previos, se hubiera empleado una gran cantidad de tiempo no solo en aprender a elaborar un buen \textit{frontend} desde cero, sino que el objetivo principal de este trabajo de fin de grado no se centra tanto en la página web, sino en el sistema completo, pudiendo considerar la mejora del \textit{frontend} como una posible futura línea de trabajo \ref{sec:futuras-lineas-trabajo}.
            
            \paragraph{Escalabilidad}
            
                Si bien este proyecto constituye una prueba de concepto, uno de los objetivos \ref{sec:objetivos} a tener en cuenta se trataba de construir una plataforma web fácilmente extensible.

                La escalabilidad de WordPress es una de sus principales fortalezas y uno de los motivos por los que es una plataforma popular y confiable para la creación de sitios web de cualquier tamaño. Su arquitectura modular y su capacidad para trabajar con bases de datos y servidores de alta gama, la convierten en opción sólida para aquellos que buscan crear sitios web que puedan crecer y adaptarse a sus necesidades en el futuro.


            \subsubsection{WordPress y Drupal}

                También se tuvo en cuenta Drupal \cite{drupal}, otro CMS \textit{open-source} que se encuentra entre los más populares y que, a pesar de no ser tan conocido como WordPress, también es una opción muy interesante para la creación de páginas web.

                \paragraph{Facilidad de uso, personalización y escalabilidad}
                
                    Observando la tabla anterior se puede apreciar varios motivos por los que se ha optado usar Wordpress en lugar de Drupal, pero todos ellos relacionados al mismo punto: \textbf{la complejidad de Drupal no se ajusta a las necesidades de este proyecto}.
    
                    Si bien puede ser una idea interesante a largo plazo y en un hipotético caso de evolución de la plataforma web a un proyecto superior, para una prueba de concepto como la que se está llevando a cabo, se debe optar por una herramienta más sencilla y fácil de usar.
    
                    \newpage
    

        \subsection{Base de Datos}
        
            \subsubsection{SQLite vs MySQL}

                SQLite es un RDBMS \cite{sqlite} que se caracteriza por ser ligero, rápido y fácil de usar; se trata de una buena opción para proyectos pequeños, pero no es recomendable para proyectos de gran envergadura, ya que no es capaz de manejar grandes cantidades de datos.
                
                Por otro lado, MySQL también es otro RDBMS \cite{mysql} que se caracteriza por ser rápido, seguro y fácil de usar; al contrario que con SQLite, MySQL sí es recomendable para proyectos de gran envergadura, ya que es capaz de manejar grandes cantidades de datos.

                \begin{table}[h]
                    \centering
                    
                    \begin{tabular}{|>{\centering\arraybackslash}m{4cm}|>{\centering\arraybackslash}m{5cm}|>{\centering\arraybackslash}m{5cm}|}
                        \hline
                        \textbf{Características} & \textbf{SQLite} & \textbf{MySQL} \\
                        \hline
                        \hline
                        \textbf{Tipo de base de datos} & Relacional, integrada & Relacional, cliente-servidor \\
                        \hline
                        \textbf{Gestión de usuarios} & Debe programarse & Integrada \\
                        \hline
                        \textbf{Escalabilidad} & Limitada: por su naturaleza integrada & Escalable: en función del hardware disponible \\
                        \hline
                        \textbf{Confiabilidad} & Menor capacidad de recuperación de datos & Mayor capacidad de recuperación de datos \\
                        \hline
                        \textbf{Flexibilidad} & Limitada en cuanto a configuración de memoria & Mayor flexibilidad en configuración \\
                        \hline
                        \textbf{Seguridad} & Limitada: no ofrece encriptación de datos & Mayor seguridad: ofrece encriptación de datos \\
                        \hline
                    \end{tabular}
                        
                    \caption{Comparativa entre SQLite y MySQL.}
                    \label{tabla:mysql-vs-sqlite}
                \end{table}

                En función de la naturaleza del proyecto a realizar, se ha considerado usar MySQl debido a las siguientes razones:
                    
                \paragraph{Gestión de usuarios integrada}
                    
                    Al tratarse de una plataforma web que requiere la gestión de usuarios, MySQL ofrece una gestión de usuarios integrada, lo que simplifica y agiliza el proceso de gestión de usuarios en comparación con SQLite.
                
                \paragraph{Escalabilidad}
                    
                    MySQL ofrece una mayor escalabilidad que SQLite debido a su arquitectura cliente-servidor. Esto significa que puede manejar grandes cantidades de datos y muchos usuarios de manera simultánea, lo que es importante en un entorno en el que se espera que varios usuarios se conecten y utilicen la plataforma al mismo tiempo.

                \paragraph{Confiabilidad}
                    
                    MySQL ofrece una mayor capacidad de recuperación de datos en comparación con SQLite. Esto es importante en un proyecto que busca ser utilizado por varias personas, ya que cualquier pérdida de datos podría tener un impacto significativo.

                \paragraph{Seguridad}
                    
                    MySQL ofrece mayor seguridad que SQLite al contar con encriptación de datos. Esto es importante en un proyecto que busca proteger los datos de los usuarios y prevenir posibles ciberataques.

                Además, también se ha tenido en cuenta su integración con WordPress, herramienta sleccionada para la página web de la plataforma, ya que WordPress cuenta con integración inmediata con MySQL desde su instalación.    


        \subsection{Alojamiento}

            \subsubsection{Local}

                Se planteó el uso de un servidor local para el alojamiento de la plataforma web, ya que permite realizar pruebas de forma rápida y sencilla, sin necesidad de contratar un servidor externo.

                Sin embargo, se descartó esta opción debido a que no se considera necesario para el desarrollo de este proyecto. El uso de un servidor local se planteó como una opción para el alojamiento de la plataforma web, ya que permite realizar pruebas de forma rápida y sencilla, sin necesidad de contratar un servidor externo.
                
            \subsubsection{Linode}

                Linode es una empresa que ofrece servicios de alojamiento en la nube. Ofrece servidores virtuales privados (VPS) con diferentes características y precios, lo que permite a los usuarios elegir el plan que mejor se adapte a sus necesidades. Además, Linode cuenta con una interfaz de usuario intuitiva y fácil de usar, así como con una amplia documentación y soporte técnico para ayudar a los usuarios a configurar y administrar sus servidores.

                \begin{figure}[htbp!]
                    \centering

                    \includegraphics[width=0.15\textwidth]{images/Logos/linode.png}
                    \caption{Logo de Linode.}

                    \label{fig:linode-logo}
                \end{figure}


                Este proyecto requiere la gestión de contenedores Docker, por lo que se buscaba un servicio que integrara Kubernetes para facilitar la gestión de los mismos.

                \paragraph{Linode Kubernetes Engine (LKE)}

                    Linode ofrece un servicio de Kubernetes administrado que permite a los usuarios implementar y administrar clústeres de Kubernetes en la nube. Este servicio se encarga de la configuración, el aprovisionamiento y la administración de los nodos del clúster, lo que permite a los usuarios centrarse en el desarrollo de sus aplicaciones en lugar de en la gestión de la infraestructura subyacente. Además, Linode Kubernetes Engine (LKE) es compatible con herramientas y servicios de terceros, lo que permite a los usuarios integrar fácilmente sus aplicaciones con otros servicios en la nube.

            \subsubsection{Amazon Web Service (AWS)}

                Amazon Web Services (AWS) es una plataforma de servicios en la nube que ofrece una amplia gama de servicios, incluyendo almacenamiento, bases de datos, redes, análisis, aprendizaje automático, etc. Esta plataforma permite a los usuarios crear aplicaciones escalables y de alta disponibilidad sin necesidad de invertir en hardware o infraestructura física.

                \begin{figure}[htbp!]
                    \centering

                    \includegraphics[width=0.15\textwidth]{images/Logos/aws.png}
                    \caption{Logo de AWS.}

                    \label{fig:aws-logo}
                \end{figure}

                Este proyecto requiere la gestión de contenedores Docker, por lo que se buscaba un servicio que integrara Kubernetes para facilitar la gestión de los mismos.

                \paragraph{Amazon Elastic Kubernetes Service (EKS)}

                    Amazon Elastic Kubernetes Service (EKS) es un servicio de Kubernetes administrado que permite a los usuarios implementar y administrar clústeres de Kubernetes en la nube. Este servicio se encarga de la configuración, el aprovisionamiento y la administración de los nodos del clúster, lo que permite a los usuarios centrarse en el desarrollo de sus aplicaciones en lugar de en la gestión de la infraestructura subyacente. Además, Amazon EKS es compatible con herramientas y servicios de terceros, lo que permite a los usuarios integrar fácilmente sus aplicaciones con otros servicios en la nube.

            \subsubsection{Google Cloud Platform (GCP)}

                Google Cloud Platform (GCP) es una plataforma de servicios en la nube que ofrece una amplia gama de servicios, incluyendo almacenamiento, bases de datos, redes, análisis, aprendizaje automático, etc. Esta plataforma permite a los usuarios crear aplicaciones escalables y de alta disponibilidad sin necesidad de invertir en hardware o infraestructura física.

                \begin{figure}[htbp!]
                    \centering

                    \includegraphics[width=0.15\textwidth]{images/Logos/gcp.png}
                    \caption{Logo de GCP.}

                    \label{fig:gcp-logo}
                \end{figure}

                Este proyecto requiere la gestión de contenedores Docker, por lo que se buscaba un servicio que integrara Kubernetes para facilitar la gestión de los mismos.

                % Esta plataforma de Google ofrece una gran cantidad de servicios en la nube con la que poder alojar distintos proyectos en función de sus necesidades.
                
                % Se planteó el uso de esta herramienta para el alojamiento de la plataforma web, porque ofrece un saldo en euros equivalentes a 3 meses de uso sus servicios, por lo que resulta muy interesante para el desarrollo de este proyecto.
                
                % Algunos de los servicios más relevantes son los siguientes:

                \paragraph{Google Kubernetes Engine (GKE)}

                    Este servicio de Google permite ejecutar aplicaciones en contenedores de forma sencilla y rápida, sin necesidad de gestionar la infraestructura subyacente. Además, permite escalar automáticamente las aplicaciones en función de la demanda, lo que resulta muy interesante para el desarrollo de este proyecto.

                    Este servicio se planteó como una opción para el alojamiento de la plataforma web, ya que permite ejecutar aplicaciones en contenedores de forma sencilla y rápida, sin necesidad de gestionar la infraestructura subyacente. Además, permite escalar automáticamente las aplicaciones en función de la demanda, lo que resulta muy interesante para el desarrollo de este proyecto.

                    Sin embargo, se descartó esta opción debido a que no se considera necesario para el desarrollo de este proyecto. GKE es un servicio que se encuentra en fase beta, por lo que no se considera adecuado para un proyecto de estas características, ya que se trata de una prueba de concepto y no se espera que la plataforma web vaya a ser utilizada por un gran número de usuarios.

                \paragraph{Cloud Run}

                    Este servicio de Google permite ejecutar contenedores Docker de forma sencilla y rápida, sin necesidad de gestionar la infraestructura subyacente. Además, permite escalar automáticamente las aplicaciones en función de la demanda, lo que resulta muy interesante para el desarrollo de este proyecto.

                    \begin{figure}
                        \centering

                        \includegraphics[width=0.15\textwidth]{images/Logos/cloud-run.png}
                        \caption{Logo de Cloud Run.}
                        
                        \label{fig:cloud-run-logo}
                    \end{figure}

                    Este servicio se planteó como una opción para el alojamiento de la plataforma web, ya que permite ejecutar contenedores Docker de forma sencilla y rápida, sin necesidad de gestionar la infraestructura subyacente. Además, permite escalar automáticamente las aplicaciones en función de la demanda, lo que resulta muy interesante para el desarrollo de este proyecto.

                    Sin embargo, se descartó esta opción debido a que no se considera necesario para el desarrollo de este proyecto. Cloud Run es un servicio que se encuentra en fase beta, por lo que no se considera adecuado para un proyecto de estas características, ya que se trata de una prueba de concepto y no se espera que la plataforma web vaya a ser utilizada por un gran número de usuarios.

                    Este planteamiento sería más adecuado para un estado más avanzado de la plataforma web, para un uso real y con un número de usuarios considerable y consistente; mientras que para el caso actual, se requiere una gestión más simple y sencilla para los contenedores Docker.

                \paragraph{Computing Engine}

                    Este servicio de Google permite crear máquinas virtuales en la nube, lo que resulta muy interesante para el desarrollo de este proyecto.

                    Este servicio se planteó como una opción para el alojamiento de la plataforma web, ya que permite crear máquinas virtuales en la nube, lo que resulta muy interesante para el desarrollo de este proyecto.

                    Sin embargo, se descartó esta opción debido a que no se considera necesario para el desarrollo de este proyecto. Computing Engine es un servicio que se encuentra en fase beta, por lo que no se considera adecuado para un proyecto de estas características, ya que se trata de una prueba de concepto y no se espera que la plataforma web vaya a ser utilizada por un gran número de usuarios.

                    Este planteamiento sería más adecuado para un estado más avanzado de la plataforma web, para un uso real y con un número de usuarios considerable y consistente; mientras que para el caso actual, se requiere una gestión más simple y sencilla para los contenedores Docker.

                    \cleardoublepage



\chapter{Diseño y desarrollo de la plataforma}
    
    \section{Ingeniería de Requisitos}
        \label{cap:ingenieria-requisitos}
        
        
        
        El único actor de la plataforma será el usuario que haga uso de ella, esté o no registrado en la misma, por lo que no se especificará este detalle en los Casos de Uso \ref{sec:casos-uso} puesto que siempre será el mismo.
        
        
        \subsection{Casos de uso}
            \label{sec:casos-uso}
            
            \begin{figure}[h]
                \centering
                \includegraphics[scale=0.125]{images/Diagramas/Casos de uso.png}
                \caption{Casos de uso}
                \label{fig:casos-uso}
            \end{figure}
            
            \newpage
        
        
        \subsection{Requisitos funcionales}
            \label{sec:requisitos-funcionales}
            
            Los casos de uso permiten identificar los requisitos funcionales necesarios para satisfacer las necesidades del usuario y asegurarse de que el sistema cumpla con sus objetivos y expectativas. Teniendo eso en cuenta, los casos de uso describen cómo interactúan los usuarios con el sistema y qué acciones o funcionalidades deben estar disponibles en cada situación.
            
            Para este proyecto, se han definido los siguientes requisitos en función de los casos de uso mencionados en la sección anterior \ref{sec:casos-uso}:
            
            
            \subsubsection{Usuarios}
            
                Requisitos relacionados con la gestión de usuarios y sus datos.
                
                \begin{table}[!htbp]
                    \centering
                    \begin{tabular}{|c|c|}
                        \hline
                        \textbf{RF - 01} & \textbf{Registro de usuarios} \\
                        \hline
                        \multicolumn{2}{|p{15cm}|}{
                            El sistema debe permitir que los usuarios se registren y creen una cuenta en la plataforma para poder usar las \textit{sandboxes}.
                        } \\
                        \hline
                        \multicolumn{2}{|p{15cm}|}{
                            \begin{itemize}
                                \item Nombre de usuario.
                                \item Dirección de correo electrónico.
                                \item Contraseña.
                            \end{itemize}
                            } \\
                        \hline
                    \end{tabular}
                    \label{tab:RF1}
                \end{table}
                
                \begin{table}[!htbp]
                    \centering
                    \begin{tabular}{|c|c|}
                        \hline
                        \textbf{RF - 02} & \textbf{Inicio de sesión} \\
                        \hline
                        \multicolumn{2}{|p{15cm}|}{
                            El sistema debe permitir que los usuarios registrados puedan iniciar sesión para poder acceder al uso de las \textit{sandboxes}.
                        } \\
                        \hline
                    \end{tabular}
                    \label{tab:RF2}
                \end{table}
                
                \begin{table}[!htbp]
                    \centering
                    \begin{tabular}{|c|c|}
                        \hline
                        \textbf{RF - 03} & \textbf{Validación de los datos de registro} \\
                        \hline
                        \multicolumn{2}{|p{15cm}|}{
                            El sistema debe comprobar que los datos de registro de un usuario sean válidos.
                        } \\
                        \hline
                        \multicolumn{2}{|p{15cm}|}{
                            \begin{itemize}
                                \item No se registró un usuario con el mismo nombre de usuario ni correo electrónico.
                                \item Si alguno de los datos no es válido (error), se informará al usuario.
                            \end{itemize}
                            } \\
                        \hline
                    \end{tabular}
                    \label{tab:RF3}
                \end{table}
            
            
            \subsubsection{Contenido}
            
                Requisitos relacionados con la gestión de contenido y sus características.
                
                \begin{table}[!htbp]
                    \centering
                    \begin{tabular}{|c|c|}
                        \hline
                        \textbf{RF - 04} & \textbf{Consulta de contenido} \\
                        \hline
                        \multicolumn{2}{|p{15cm}|}{
                            El sistema debe proporcionar contenido relevante que permita a los usuarios adquirir los conocimientos necesarios para realizar las pruebas.
                        } \\
                        \hline
                    \end{tabular}
                    \label{tab:RF4}
                \end{table}
                
                \newpage
            
            
            \subsubsection{\textit{Sandboxes}}
            
                Requisitos relacionados con la gestión de los entornos virtualizados.
                
                \begin{table}[!htbp]
                    \centering
                    \begin{tabular}{|c|c|}
                        \hline
                        \textbf{RF - 05} & \textbf{Creación de \textit{sandboxes}} \\
                        \hline
                        \multicolumn{2}{|p{15cm}|}{
                            El sistema debe permitir que los usuarios seleccionen e inicien un laboratorio de pruebas que deseen realizar.
                        } \\
                        \hline
                        \multicolumn{2}{|p{15cm}|}{
                            \begin{itemize}
                                \item Si no es posible iniciar un laboratorio (error), se informará al usuario.
                            \end{itemize}
                            } \\
                        \hline
                    \end{tabular}
                    \label{tab:RF5}
                \end{table}
                
                \begin{table}[!htbp]
                    \centering
                    \begin{tabular}{|c|c|}
                        \hline
                        \textbf{RF - 06} & \textbf{Tiempo de vida de las \textit{sandboxes}} \\
                        \hline
                        \multicolumn{2}{|p{15cm}|}{
                            El sistema debe destruir automáticamente un laboratorio iniciado una vez que se haya superado un tiempo de vida definido.
                        } \\
                        \hline
                        \multicolumn{2}{|p{15cm}|}{
                            \begin{itemize}
                                \item Definir un tiempo de vida fijo para las \textit{sandboxes}.
                            \end{itemize}
                            } \\
                        \hline
                    \end{tabular}
                    \label{tab:RF6}
                \end{table}
                
                \begin{table}[!htbp]
                    \centering
                    \begin{tabular}{|c|c|}
                        \hline
                        \textbf{RF - 07} & \textbf{Número máximo de \textit{sandboxes} iniciadas} \\
                        \hline
                        \multicolumn{2}{|p{15cm}|}{
                            El sistema no debe permitir el inicio de infinitos laboratorios.
                        } \\
                        \hline
                        \multicolumn{2}{|p{15cm}|}{
                            \begin{itemize}
                                \item Definir un número máximo de instancias posibles.
                            \end{itemize}
                            } \\
                        \hline
                    \end{tabular}
                    \label{tab:RF7}
                \end{table}
                
                \begin{table}[!htbp]
                    \centering
                    \begin{tabular}{|c|c|}
                        \hline
                        \textbf{RF - 08} & \textbf{Uso de una \textit{sandbox} iniciada} \\
                        \hline
                        \multicolumn{2}{|p{15cm}|}{
                            El sistema debe permitir que los usuarios puedan conectarse a un laboratorio una vez este se haya iniciado.
                        } \\
                        \hline
                        \multicolumn{2}{|p{15cm}|}{
                            \begin{itemize}
                                \item Proporcionar los datos de conexión a un laboratorio al usuario.
                                \item La conexión se realizará por el propio usuario a través de SSH.
                            \end{itemize}
                            } \\
                        \hline
                    \end{tabular}
                    \label{tab:RF8}
                \end{table}
                
                \begin{table}[!htbp]
                    \centering
                    \begin{tabular}{|c|c|}
                        \hline
                        \textbf{RF - 09} & \textbf{Destrucción de \textit{sandboxes}} \\
                        \hline
                        \multicolumn{2}{|p{15cm}|}{
                            El sistema debe permitir que los usuarios puedan apagar una \textit{sandbox}, lo que equivale a destruirla manualmente (en lugar de esperar a que se cumpla su tiempo de vida).
                        } \\
                        \hline
                        \multicolumn{2}{|p{15cm}|}{
                            \begin{itemize}
                                \item Si no es posible destruir un laboratorio (error), se informará al usuario.
                                \item Un administrador podrá acceder a registros de uso de los laboratorios.
                            \end{itemize}
                            } \\
                        \hline
                    \end{tabular}
                    \label{tab:RF09}
                \end{table}
                
                \newpage
        
        
        \subsection{Requisitos no funcionales}
            \label{sec:requisitos-nofuncionales}
            
            Por otro lado, los requisitos no funcionales describen las cualidades o atributos que debe tener un sistema, enfocándose en cómo hace el sistema lo que hace; es decir, cómo se comporta en términos de calidad.
            
            Para este proyecto, se han definido los siguientes requisitos en función de los requisitos mencionados en la sección anterior \ref{sec:requisitos-funcionales} y los objetivos de la plataforma:
            
            \begin{table}[!htbp]
                \centering
                \begin{tabular}{|c|c|}
                    \hline
                    \textbf{RNF - 01} & \textbf{Privacidad de los datos} \\
                    \hline
                    \multicolumn{2}{|p{15cm}|}{
                        El sistema debe garantizar la seguridad y privacidad de los datos de los usuarios.
                    } \\
                    \hline
                    \multicolumn{2}{|p{15cm}|}{
                        \begin{itemize}
                            \item Aplicar una política de cifrado para las contraseñas.
                        \end{itemize}
                        } \\
                    \hline
                \end{tabular}
                \label{tab:RNF1}
            \end{table}
            
            \begin{table}[!htbp]
                \centering
                \begin{tabular}{|c|c|}
                    \hline
                    \textbf{RNF - 02} & \textbf{Aislamiento de las \textit{sandboxes}} \\
                    \hline
                    \multicolumn{2}{|p{15cm}|}{
                        El sistema debe garantizar la seguridad y aislamiento de los laboratorios de prueba de la plataforma.
                    } \\
                    \hline
                    \multicolumn{2}{|p{15cm}|}{
                        \begin{itemize}
                            \item Aislar completamente los laboratorios sin afectar al resto de la plataforma.
                        \end{itemize}
                        } \\
                    \hline
                \end{tabular}
                \label{tab:RNF2}
            \end{table}
            
            \begin{table}[!htbp]
                \centering
                \begin{tabular}{|c|c|}
                    \hline
                    \textbf{RNF - 03} & \textbf{Seguridad de la plataforma} \\
                    \hline
                    \multicolumn{2}{|p{15cm}|}{
                        El sistema debe implementar medidas de seguridad para prevenir ataques externos o internos a la plataforma.
                    } \\
                    \hline
                    \multicolumn{2}{|p{15cm}|}{
                        \begin{itemize}
                            \item Medidas de autenticación y autorización para garantizar el acceso solo a usuarios autorizados.
                        \end{itemize}
                        } \\
                    \hline
                \end{tabular}
                \label{tab:RNF3}
            \end{table}
            
            \begin{table}[!htbp]
                \centering
                \begin{tabular}{|c|c|}
                    \hline
                    \textbf{RNF - 04} & \textbf{Accesibilidad} \\
                    \hline
                    \multicolumn{2}{|p{15cm}|}{
                        La plataforma debe ser accesible para cualquier usuario, independientemente de su nivel de experiencia en ciberseguridad.
                    } \\
                    \hline
                    \multicolumn{2}{|p{15cm}|}{
                        \begin{itemize}
                            \item Contenido claro y accesible.
                            \item Laboratorios bien estructurados.
                        \end{itemize}
                        } \\
                    \hline
                \end{tabular}
                \label{tab:RNF4}
            \end{table}
            
            \begin{table}[!htbp]
                \centering
                \begin{tabular}{|c|c|}
                    \hline
                    \textbf{RNF - 05} & \textbf{Usabilidad} \\
                    \hline
                    \multicolumn{2}{|p{15cm}|}{
                        La plataforma debe ser fácil de usar y accesible para cualquier usuario, independientemente de su nivel de experiencia en ciberseguridad.
                    } \\
                    \hline
                    \multicolumn{2}{|p{15cm}|}{
                        \begin{itemize}
                            \item Interfaz de usuario intuitiva, clara y organizada.
                        \end{itemize}
                        } \\
                    \hline
                \end{tabular}
                \label{tab:RNF5}
            \end{table}
            
            \begin{table}[H]
                \centering
                \begin{tabular}{|c|c|}
                    \hline
                    \textbf{RNF - 06} & \textbf{Extensibilidad} \\
                    \hline
                    \multicolumn{2}{|p{15cm}|}{
                        La plataforma debe ser fácilmente extensible, permitiendo la futura integración de más contenido y laboratorios de pruebas.
                    } \\
                    \hline
                    \multicolumn{2}{|p{15cm}|}{
                        \begin{itemize}
                            \item Contenedores Docker y páginas de documentación en el sitio web.
                        \end{itemize}
                        } \\
                    \hline
                \end{tabular}
                \label{tab:RNF6}
            \end{table}
            
            \cleardoublepage
    
    
    \section{Arquitectura}
        \label{sec:arquitectura}
        
        \begin{figure}[h]
            \centering
            \includegraphics[scale=0.20]{images/Diagramas/Arquitectura.png}
            \caption{Arquitectura}
            \label{fig:arquitectura}
        \end{figure}

        Los componentes de la plataforma podrían dividirse en 4 partes fundamentales: el front-end, el back-end, la base de datos y los contenedores Docker (laboratorios). Todos estos están desplegados en un servidor que utiliza LAMP.

        \subsection{\textit{Front-end}: WordPress}

            El front-end es la parte de la plataforma que interactúa con el usuario. En este caso, se ha desarrollado una aplicación web que permite al usuario acceder a la plataforma y realizar las acciones que se describen en la sección \ref{sec:casos-uso}.
            
            Esta plataforma web se ha desarrollado utilizando el gestor de contenidos WordPress, ya que permite crear sitios web de forma rápida y sencilla, sin necesidad de tener conocimientos avanzados de programación. Además, cuenta con una gran comunidad de desarrolladores que crean y mantienen plugins y temas para extender las funcionalidades de la plataforma.


        \subsection{\textit{Back-end}: \texttt{functions.php}}
        
            El back-end es la parte de la plataforma que se encarga de procesar las peticiones del usuario y de gestionar los recursos de la plataforma. En este caso, se encarga de gestionar las peticiones del usuario y de comunicarse con la base de datos y los contenedores Docker.
            
            Se ha usado el fichero \textit{functions.php} de WordPress para implementar el back-end de la plataforma, ya que permite extender las funcionalidades de WordPress de forma sencilla. Se ejecuta en cada petición que se realiza a la plataforma, por lo que se ha usado para implementar las funcionalidades necesarias para la gestión y configuración de los laboratorios de pruebas ejecutados por los usuarios.

            Además, se ha añadido una tarea de cron con la que comprobar si algún laboratorio de pruebas superó su tiempo de vida (1 hora). Esta tarea se ejecuta cada minuto y destruye dichos laboratorios, liberando recursos en el sistema.


        \subsection{Base de datos: MySQL}

            La base de datos es la parte de la plataforma que se encarga de almacenar la información de los usuarios y de los laboratorios de pruebas. En este caso, se ha usado para almacenar la información de los usuarios registrados en la plataforma y de los laboratorios de pruebas que han sido creados por dichos usuarios.
            
            Se ha usado MySQL como gestor de base de datos, ya que es un sistema de gestión de bases de datos relacional, de código abierto y muy popular. Además, es compatible con WordPress, por lo que se puede acceder a la base de datos desde el fichero \textit{functions.php}.
        
        
        \subsection{Contenedores Docker: laboratorios de pruebas}

            Los contenedores Docker son la parte de la plataforma que se encarga de ejecutar los laboratorios de pruebas. En este caso, se ha usado para ejecutar los laboratorios de pruebas que han sido creados por los usuarios.
            
            Se ha usado Docker para ejecutar los laboratorios de pruebas, ya que permite ejecutar aplicaciones en contenedores de software. Estos contenedores son ligeros y portables, por lo que se pueden ejecutar en cualquier máquina que tenga Docker instalado. Además, se pueden crear imágenes de los contenedores, lo que permite crear laboratorios de pruebas personalizados y compartirlos con otros usuarios.
            
        
        \subsection{Conexión a una \textit{sandbox}}
        
        \begin{figure}[h]
            \centering
            \includegraphics[scale=0.20]{images/Diagramas/Arquitectura 1.png}
            \caption{Uso normal}
            \label{fig:conexion-sandbox}
        \end{figure}
        
        El usuario está registrado, ya que en caso contrario, no podría acceder a esta funcionalidad del sistema.
        
        \begin{itemize}
            \item El usuario que ha accedido a la plataforma quiere encender una \textit{sandbox}.
            \item Se informa al servidor de dicha acción y se consulta en la base de datos la información de dicho entorno virtualizado (ID, nombre, características de construcción, tiempo de vida...).
            \item Una vez obtenido dicha información, se procesa y se inicia la construcción de una instancia de la \textit{sandbox}.
            \item Finalmente, tras haberse construido el entorno, el usuario recibe un mensaje con los datos de conexión y puede proceder a conectarse al entorno a través de SSH.
        \end{itemize}
        
        \newpage
    
    
    \section{Modelado de actividades y transiciones}
        \label{sec:modelado-actividades-transiciones}
        
        \subsection{Tratamiento de usuarios}
        
            \begin{figure}[h]
                \centering
                \begin{subfigure}{0.45\textwidth}
                    \centering
                    \includegraphics[scale=0.15]{images/Diagramas/Actividades y transiciones 1.png}
                    \caption{Registro de un usuario}
                    \label{fig:registro-usuario}
                \end{subfigure}
                \hfill
                \begin{subfigure}{0.45\textwidth}
                    \centering
                    \includegraphics[scale=0.15]{images/Diagramas/Actividades y transiciones 2.png}
                    \caption{Inicio de sesión de un usuario}
                    \label{fig:inicio-usuario}
                \end{subfigure}
                \caption{Modelado de actividades y transiciones}
                \label{fig:actividades-transiciones}
            \end{figure}
            
            Los diagramas presentados en esta sección definen la gestión de usuarios que permite el registro e inicio de sesión de los mismos en la plataforma.
            
            El proceso de registro de un usuario mostrará inicialmente un formulario para poder obtener sus datos, debiendo verificar que dichos datos no pertenecen a un usario previamente registrado, puesto que los usuarios deben ser únicos.
            
            El proceso de inicio de sesión de un usuario mostrará un funcionamiento similar al de registro, pero esta vez, para comprobar que el usuario sí ha sido registrado anteriormente.
            
            \newpage
            
            
        \subsection{Consulta de la documentación}
            \label{sec:consulta-documentacion}
            
            \begin{figure}[h]
                \centering
                \includegraphics[scale=0.20]{images/Diagramas/Actividades y transiciones 3.png}
                \caption{Consulta de documentación}
                \label{fig:consulta-documentacion}
            \end{figure}
            
            El diagrama presentado en esta sección define el consumo de contenido de la plataforma por parte de un usuario, esté registrado o no, ya que el contenido instructivo de la plataforma se considera público, al contrario que el uso de las \textit{sandboxes} que requerirá un registro por parte del usuario.
            
            El proceso de consulta de documentación es bastante simple: un usuario puede consumir contenido de forma continua, pero al estar dividido por conceptos, necesitará cambiar de ubicación dentro de la plataforma para poder seguir accediendo a contenido nuevo.
            
            \newpage
            
            
        \subsection{Tratamiento de \textit{sandboxes}}
        
            \begin{figure}[h]
                \centering
                
                \begin{subfigure}{0.45\textwidth}
                    \centering
                    
                    \includegraphics[scale=0.115]{images/Diagramas/Actividades y transiciones 4.png}
                    
                    \caption{Creación de una \textit{sandbox}}
                    \label{fig:creacion-sandbox}
                \end{subfigure}

                \hfill
                
                \begin{subfigure}{0.45\textwidth}
                    \centering
                    
                    \includegraphics[scale=0.10]{images/Diagramas/Actividades y transiciones 5.png}
                    
                    \caption{Destrucción de una \textit{sandbox}}
                    \label{fig:destruccion-sandbox}
                \end{subfigure}
                
                \caption{Tratamiento de \textit{sandboxes}}
                \label{fig:tratamiento-sandboxes}
            \end{figure}
            
            Los diagramas presentados en esta sección definen la gestión de entornos virtualizados de la plataforma.
            
            El proceso de creación de una \textit{sandbox} iniciará pulsando un botón, preferiblemente ubicado al final de un contenido instructivo descrito anteriormente en la \autoref{sec:consulta-documentacion}, debiendo verificar que es posible crear dicha \textit{sandbox}.
            
            El proceso de destrucción de una \textit{sandbox} seguirá el mismo proceso que el anterior, pero realizará la opción opuesta.
            
            \newpage
        
        
    \section{Catálogo de conceptos}
        \label{sec:catalogo}

        \subsection{Introducción básica a Linux y Bash scripting}

            
    
        \subsection{Escaneo de puertos y enumeración de servicios}
        
            El escaneo de puertos es una técnica donde se envían paquetes a un rango de puertos de un sistema con el objetivo de encontrar aquellos que están abiertos y escuchando conexiones; esta técnica se utiliza para descubrir posibles vulnerabilidades en sistemas informáticos y redes.
            
            La enumeración de servicios, por otro lado, es el proceso de identificación y catalogación de los servicios que se están ejecutando en un sistema informático. Los servicios son programas que se ejecutan en segundo plano en un sistema informático y que se utilizan para proporcionar una funcionalidad específica.
            
            
        \subsection{Inyección SQL}
            
            La inyección SQL es una técnica de ataque que se utiliza para explotar vulnerabilidades en las aplicaciones web que interactúan con bases de datos. Esta técnica se basa en la inserción de código SQL malintencionado en los campos de entrada de una aplicación, con el fin de obtener acceso no autorizado a los datos almacenados en la base de datos o incluso tomar el control del servidor.
            
            La inyección SQL se aprovecha de la falta de validación o sanitización de los datos de entrada que son utilizados en las consultas SQL. Cuando los datos de entrada no son validados correctamente, los atacantes pueden introducir código SQL malintencionado que es ejecutado por la aplicación. Esto puede permitir a los atacantes extraer información confidencial, realizar modificaciones en la base de datos o incluso tomar el control total del servidor.
        
        
        \subsection{Ataques de fuerza bruta}
            
            Los ataques de fuerza bruta son una técnica de hacking en la que un atacante intenta adivinar una contraseña o credencial de acceso al probar diferentes combinaciones de contraseñas hasta encontrar la correcta. Estos ataques suelen ser automatizados y pueden ser extremadamente efectivos si el atacante tiene suficiente tiempo y recursos.
        
        
        \subsection{Escalada de privilegios}
            
            La escalada de privilegios es una técnica de hacking que se utiliza para obtener acceso a sistemas o recursos que normalmente estarían restringidos por permisos de seguridad. Esta técnica se utiliza cuando un atacante ya ha obtenido acceso a un sistema con un conjunto limitado de permisos, pero necesita obtener permisos adicionales para llevar a cabo acciones malintencionadas.
            
            En este tipo de ataque, el atacante busca explotar vulnerabilidades en el sistema operativo o en las aplicaciones instaladas para obtener permisos de administrador o de otro usuario privilegiado. Una vez que se obtienen estos permisos, el atacante puede tener acceso a información confidencial, instalar malware o tomar el control total del sistema.
        
        
        \subsection{Ataques de \textit{buffer overflow}}
            
            Un ataque de \textit{buffer overflow} es una técnica de explotación de vulnerabilidades en la memoria de un programa que permite a un atacante ejecutar código malicioso en un sistema. El ataque se produce cuando un programa intenta escribir datos en un buffer de memoria, pero la cantidad de datos escritos excede la capacidad del buffer, permitiendo al atacante escribir un bloque de código malicioso en el buffer y sobreescribir la dirección de retorno de la función, para que apunte al código malicioso en lugar de volver a la función llamada originalmente.
            
            Este tipo de ataque puede ser especialmente peligroso si el programa original que se está ejecutando tiene privilegios elevados, ya que el código malicioso puede ejecutarse con esos mismos privilegios. Por lo tanto, los ataques de \textit{buffer overflow} pueden ser utilizados para obtener acceso no autorizado a un sistema o para ejecutar código malicioso en un sistema.
        
        
        \subsection{Análisis de tráfico}
            
            El análisis de tráfico es una técnica de ciberseguridad que implica la monitorización y el examen de los datos que fluyen a través de una red de comunicaciones. Este análisis puede incluir la recopilación de información sobre el origen y el destino de los datos, el tipo de protocolo utilizado, la cantidad de datos transmitidos y otros metadatos relacionados con el tráfico; puede ser utilizado para detectar patrones de actividad sospechosa en una red, como la transferencia de grandes cantidades de datos o la comunicación con direcciones IP o puertos inusuales. También puede ser utilizado para identificar amenazas específicas, como ataques de denegación de servicio o intentos de acceso no autorizado a sistemas.
            
            El análisis de tráfico puede llevarse a cabo utilizando herramientas de software especializadas que pueden recopilar y analizar los datos de la red en tiempo real. Estas herramientas pueden generar alertas automáticas cuando se detecta actividad sospechosa y proporcionar informes detallados sobre el tráfico de red.
            
            Se debe tener en cuenta que el análisis de tráfico puede presentar desafíos legales y de privacidad, ya que puede involucrar la monitorización de datos confidenciales, por lo que es importante asegurarse de que cualquier análisis de tráfico se lleve a cabo de manera legal y ética y que se respeten las políticas de privacidad y seguridad de la organización.
        
        \subsection{Análisis de metadatos}
            
            El análisis de metadatos es una técnica de ciberseguridad que implica analizar los metadatos asociados con diferentes tipos de archivos. Los metadatos son información adicional que se almacena junto con los archivos, como la fecha de creación, la ubicación, el tipo de archivo, el autor y otros detalles. Este análisis se utiliza a menudo para identificar posibles amenazas de ciberseguridad, como la presencia de archivos maliciosos o la detección de patrones de actividad sospechosa.
            
            \cleardoublepage


\chapter{Problemas encontrados}

    \section{Gestión y automatización local de contenedores Docker}

    
    
\chapter{Resultados}

    \section{Cumplimiento de los requisitos}
    
    \section{Conclusiones}
    
    \section{Futuras líneas de trabajo}
        \label{sec:futuras-lineas-trabajo}
        
        Expandir el contenido de la plataforma mediante más documentación sobre conceptos adicionales, junto a nuevas pruebas virtualizadas.
        
        Elaboración de Módulos o Rutas, que se definirían como un grupo de conceptos a tratar en un orden determinado, para obtener conocimiento sobre un tema concreto.
        
        El enfoque inicial de la plataforma era el pentesting, pero si se aplicara lo anterior, podría establecerse una naturaleza genérica para abarcar muchos más conceptos de distitnas áreas de la seguridad: Administración de Sistemas, Desarrollo y Análisis de Malware... dando lugar a una plataforma más eficiente y parecida a otras ya existentes, pero manteniendo la diferencia que se fomentaba al inicio del proyecto en la sección Objetivos \ref{sec:objetivos}: una plataforma ligera, \textit{open-source} y fácilmente extensible.

        \cleardoublepage
        


\chapter{Diario}
    \label{cap:diario}

    Cada sección representa los cambios realizados hasta cada reunión.

    \section{7ª reunión: 12/04/2023}
    
        \textbf{Se crearon los capítulos: Diario \ref{cap:diario} y Borrador \ref{cap:investigacion-previa}} \\
        El primero, para incluir un historial de mi progreso; el segundo, para subir contenido WIP (\textit{work in progress}) al documento.
        
        \textbf{Se modificó el diagrama: casos de uso \ref{fig:casos-uso}} \\
        Se aplicó un estilo estándar y se mejoraron las dependencias.
        
        \textbf{Se creó la sección: Arquitectura \ref{sec:arquitectura}} \\
        Se incluyeron nuevos diagramas y contenido descriptivo sobre la arquitectura del proyecto.
        
        \textbf{Se creó la sección: Modelado de Actividades y Transacciones \ref{sec:modelado-actividades-transiciones}} \\
        Se incluyeron nuevos diagramas y contenido descriptivo de los procesos del proyecto.
        
        \textbf{Se modificó la sección: Ingeniería de Requisitos \ref{cap:ingenieria-requisitos}} \\
        Se descompusieron los requisitos anteriores en requisitos más atómicos y se han añadido nuevos requisitos no funcionales.
    
        \textbf{Se creó la sección: Estructura global del proyecto \ref{sec:estructura-global}} \\
        Se incluyó una descripción de lo que se plantea realizar con este TFG separándolo en 3 pilares fundamentales.
    
        \textbf{Se creó la sección: Proceso Creativo \ref{sec:proceso-creativo}} \\
        Se incluyó una descripción de distintos planteamientos sobre la plataforma web que tuvieron lugar durante la investigación previa del TFG. \\
        El nombre de esta sección puede variar, pero al ser un borrador, se elegió algo \textit{llamativo}.
    
        \textbf{Se inició la creación de un prototipo de la plataforma} \\
        Se está configurando una página WordPress (WIP).\\
        Se está practicando con contenedores de Docker (WIP)\\ 
        Se está tratando de levantar un contenedor a través de la página (WIP).

        \newpage


    \section{8ª reunión: 03/05/2023}

        \textbf{Se añadió una página: contraportada} \\
        Se unió el documento PDF de la contraportada del campus virtual al final del documento.

        \textbf{Se modificó la bibliografía: enlace de la ADA \cite{articulo-ada}} \\
        Se sustituyó por una versión corta y equivalente. Ahora todos los enlaces caben en la página.

        \textbf{Se modificó un diagrama: destrucción de una \textit{sandbox} \ref{fig:destruccion-sandbox}} \\
        Se añadió un caso de control de errores y el tratamiento del tiempo de vida.

        \textbf{Se modificó una tabla: Wordpress vs Astro vs Drupal \ref{tab:wordpress-vs-astro-vs-drupal}} \\
        Se combinaron las tablas Wordpress vs Astro y Wordpress vs Drupal, de forma que aunque se mantienen sus secciones separadas (porque las comparaciones fueron distintas), solo hay una tabla como referencia.

        \textbf{Se modificó la sección: Futuras líneas de trabajo \ref{sec:futuras-lineas-trabajo}} \\
        Se reformuló el último párrafo, haciendo más hincapié en la posible expansión de la plataforma con más recursos y conceptos fuera del ámbito del pentesting, convirtiéndola no en la temática principal, sino en una de múltiples temáticas.

        \textbf{Se modificó la sección: Jupyter 5.2.2} \\
        Se estableció una conclusión para el descarte de esta tecnología en el proyecto.

        \textbf{Se modificó la sección: WordPress 1.5.2} \\
        Se añadieron nuevas referencias a la bibliografía.

        \textbf{Se creó un laboratorio: Bypass} \\
        Define el concepto de Bypass y la vulnerabilidad CVE-2017-8386 como ejemplo. \\
        El laboratorio contiene dicha vulnerabilidad para explotarla.
        
        \textbf{Se creó un laboratorio: Fuerza Bruta} \\
        Define el concepto de Fuerza Bruta, tipos de ataques más comunes y herramientas. \\
        El laboratorio contiene las herramientas \texttt{nmap} e \texttt{hydra} para que el usuario pueda obtener las credenciales del usuario administrador del entorno.
        
        \textbf{Se arregló la sección: Dependencias de contenedores Docker} \\
        La sección hacía referencia a \textit{contenedores} de Docker cuando en realidad se trataban de \textit{imágenes}. Ahora la sección cuenta con rigor y se ha optimizado el texto.

        \newpage

    \section{9ª reunión: 18/05/2023}

        \textbf{Se movió el capítulo: Borrador} \\
        El capítulo ahora es \textit{Investigación previa} (\ref{cap:investigacion-previa}).